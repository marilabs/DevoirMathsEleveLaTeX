\documentclass[11pt,a4paper]{article} 

\usepackage[a4paper,margin=2cm]{geometry}
\usepackage[T1]{fontenc}
%\usepackage{pslatex}
\usepackage[utf8]{inputenc}
\usepackage[francais]{babel} 
\usepackage{graphicx} 
\usepackage{amsmath} 
\setlength{\unitlength}{1mm}
\usepackage{enumitem}
\usepackage{cancel}
\usepackage{amssymb} % pour les ensembles NN
\usepackage{mathrsfs} % pour les hypothèses de récurrences \PP
\usepackage{fancyhdr}
\usepackage{tikz,tkz-tab}
\usepackage{fancybox}
\usepackage{titling}
\usepackage[explicit]{titlesec}
\usepackage{mathtools}

% pour avoir les références complètes aux subsubsections (e.g. 2.B.3)
\renewcommand\thesection{\arabic{section}}
\renewcommand{\thesubsection}{\thesection.\Alph{subsection}}
\renewcommand{\thesubsubsection}{\thesubsection.\arabic{subsubsection}}

% format Problème 1, Partie 1.A et Question 1.A.1
\titleformat{\section}{\normalfont\Large\bfseries}{}{0em}{#1\ \thesection}
\titleformat{\subsection}{\normalfont\large\bfseries}{}{0em}{#1\ \thesubsection}
\titleformat{\subsubsection}{\normalfont\bfseries}{}{0em}{#1\ \thesubsubsection}

% format Problème 1, Partie A et Question 1
%\titleformat{\section}{\normalfont\Large\bfseries}{}{0em}{#1\ \thesection}
%\titleformat{\subsection}{\normalfont\large\bfseries}{}{0em}{#1\ \Alph{subsection}}
%\titleformat{\subsubsection}{\normalfont\bfseries}{}{0em}{#1\ \arabic{subsubsection}}

\definecolor{Honeydew1}{rgb}{.94,1,.94}

\newcommand{\res}{\colorbox{Honeydew1}}
\newcommand{\resm}[1]{\colorbox{Honeydew1}{$\displaystyle #1$}}

\pagestyle{fancy}

\DeclareMathOperator{\NN}{\mathbb{N}}
\DeclareMathOperator{\PP}{\mathscr{P}}
\DeclareMathOperator{\RR}{\mathbb{R}}
\DeclareMathOperator{\e}{e}

\newcommand{\monNom}{Prénom Nom}
\newcommand{\maClasse}{T\textsuperscript{le}1}


\lhead{\monNom}
\chead{}
\rhead{\maClasse}

\lfoot{\jobname.tex}
\cfoot{}
\rfoot{\thepage}
\renewcommand{\headrulewidth}{0.4pt}
\renewcommand{\footrulewidth}{0.4pt}

\setlength{\droptitle}{-1.5cm}

\makeatletter
\renewcommand{\maketitle}{
  \thispagestyle{empty}
  \begin{center}
  \shadowbox{\parbox{4in}{%
     \centering%
     \textrm{\textbf{\Large \@title}}\\
     \vspace{0.2cm}
     \textrm{\large \@author}\\
     \vspace{0.2cm}
     \textrm{\large \@date}
  }} 
  \end{center}
  \null
}

\author{\monNom, \maClasse}
\title{Devoir maison x}
\date{\today} 

\begin{document}

\maketitle

\thispagestyle{fancy}

\section{Problème}

\subsection{Partie}

\subsubsection{}
Une équation
\begin{equation}
  \sqrt{x^2}=\lvert x\rvert \label{eq_abs}
\end{equation}

Une autre non numérotée:
\begin{equation*}
  \sum_{k=0}^nk^3=\left(\frac{n(n+1)}{2}\right)^2
\end{equation*}

Un système d'équations:
\begin{equation}
  \left\lbrace
  \begin{aligned}
    \binom{n}{k}      & = \frac{n!}{k!(n-k)!}                                                                       \\
    A_{m,n}           & =
    \begin{pmatrix}
      a_{1,1} & a_{1,2} & \cdots & a_{1,n} \\
      a_{2,1} & a_{2,2} & \cdots & a_{2,n} \\
      \vdots  & \vdots  & \ddots & \vdots  \\
      a_{m,1} & a_{m,2} & \cdots & a_{m,n}
    \end{pmatrix} \\
    \int_a^b f(x)\,dx & =\lim_{n\rightarrow+\infty}\frac{b-a}{n}\sum_{k=0}^{n} f\left(a+k\times\frac{b-a}{n}\right) \\
  \end{aligned}
  \right.
\end{equation}

Une référence à l'équation (\ref{eq_abs}).

Un blabla mathématique: $\forall x\in \RR^\star, \exists\, y\in \left]-1;1\right[\backslash \lbrace{0\rbrace}\, /\, x=\frac{1}{y} $

Ou encore $\forall \varepsilon>0, \exists N\in\mathbb{N}, \forall n\geqslant N, \left\vert u_n - l \right\vert < \varepsilon.$

\subsubsection{}
Un tableau de signe: \\
\begin{center}
  \begin{tikzpicture}
    \tkzTabInit[espcl=2.0]
    {$x$ / 1 ,$f(x)$ /1 }%
    {$0$ , $\alpha$ , $+\infty$}%
    \tkzTabLine{ d, -, z , +, }
  \end{tikzpicture}
\end{center}

\subsection{Partie}
\subsubsection{}
Des limites avec surlignage du résultat:

\begin{equation*}
  \left.
  \begin{aligned}
    \lim_{x\rightarrow 0}\frac{\sin x}{x} & = 1      \\
    \lim_{x\rightarrow+\infty}\ln x       & =+\infty
  \end{aligned}
  \right\rbrace \text{par produit/somme, on en déduit donc que } \resm{\lim_{x\rightarrow+\infty}f(x)=+\infty}
\end{equation*}

\subsubsection{}
\begin{enumerate}[label=(\alph*)]
  \item sous-question
  \item autre sous-question
\end{enumerate}

\subsection{Partie}

\subsubsection{}\label{q2.A.3}
Un align pour une série d'équations:
\begin{align*}
  f^\prime(x) & =(x)^\prime \\
              & = \resm{1}  \\
\end{align*}
\section{Problème}

\subsection{Partie}

\subsubsection{}
Une référence à une question \ref{q2.A.3}.

Soustraction avec ligne:

\begin{align*}
  f(x)                            & = a+\cancel{b} \\
  - \;\;\;\;\;\;\;g(x)            & = \cancel{b}-a \\
  \cline{1-3}
  f(x)-g(x)                       & = a+a          \\
  \iff \;\;\;\;\;\;\;\;\;2f(x)    & = 2\times a    \\
  \iff \;\;\;\;\;\;\;\;\;\;\;f(x) & = a            \\
\end{align*}

\subsection{Partie}
\subsubsection{}

Un alignement plus complexe:

\begin{alignat*}{3}
       &   & f(x)=\frac{\e^x-\e^{-x}}{2} & >0       \\
  \iff &   & \e^x-\e^{-x}                & >0       \\
  \iff &   & \e^x                        & >\e^{-x} \\
  \iff &   & x                           & >-x      \\
  \iff &   & 2x                          & >0       \\
  \iff &   & x                           & >0
\end{alignat*}

\subsubsection{}

Tableau de variation:\\

\begin{tikzpicture}
  \tkzTabInit[espcl=6]{$x$/1,$f'(x)$/1, $f(x)$/2}{$-\infty$,$0$,$+\infty$}% 
  \tkzTabLine{ , , +, }%
  \tkzTabVar{- / $-\infty$,R / ,+ / $+\infty$ }
  \tkzTabVal[draw]{1}{3}{0.5}{0}{0}
\end{tikzpicture}


\end{document}
