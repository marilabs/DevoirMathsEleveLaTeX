\documentclass[11pt,a4paper]{article} 

\usepackage[a4paper,margin=2cm]{geometry}
\usepackage[T1]{fontenc}
%\usepackage{pslatex}
\usepackage[utf8]{inputenc}
\usepackage[francais]{babel} 
\usepackage{graphicx} 
\usepackage{amsmath} 
\setlength{\unitlength}{1mm}
\usepackage{enumitem}
\usepackage{cancel}
\usepackage{amssymb} % pour les ensembles NN
\usepackage{mathrsfs} % pour les hypothèses de récurrences \PP
\usepackage{fancyhdr}
\usepackage{tikz,tkz-tab,tikz-3dplot,tkz-euclide}
\usetkzobj{all}
\usetikzlibrary{angles,quotes,intersections,calc,through}
\usepackage{fancybox}
\usepackage{titling}
\usepackage[explicit]{titlesec}
\usepackage{mathtools}
\usepackage{eurosym}
\usepackage{stmaryrd} % pour parallèle // \sslash

\usepackage{pgfplots}


\usepackage[lined,boxed,commentsnumbered, ruled,vlined,linesnumbered, french, onelanguage]{algorithm2e}

% pour avoir les références complètes aux subsubsections (e.g. 2.B.3)
\renewcommand\thesection{\arabic{section}}
\renewcommand{\thesubsection}{\thesection.\Alph{subsection}}
\renewcommand{\thesubsubsection}{\thesubsection.\arabic{subsubsection}}

% format Problème 1, Partie 1.A et Question 1.A.1
\titleformat{\section}{\normalfont\Large\bfseries}{}{0em}{#1\ \thesection}
\titleformat{\subsection}{\normalfont\large\bfseries}{}{0em}{#1\ \thesubsection}
\titleformat{\subsubsection}{\normalfont\bfseries}{}{0em}{#1\ \thesubsubsection}

% format Problème 1, Partie A et Question 1
%\titleformat{\section}{\normalfont\Large\bfseries}{}{0em}{#1\ \thesection}
%\titleformat{\subsection}{\normalfont\large\bfseries}{}{0em}{#1\ \Alph{subsection}}
%\titleformat{\subsubsection}{\normalfont\bfseries}{}{0em}{#1\ \arabic{subsubsection}}

\definecolor{Honeydew1}{rgb}{.94,1,.94}

\newcommand{\res}{\colorbox{Honeydew1}}
\newcommand{\resm}[1]{\colorbox{Honeydew1}{$\displaystyle #1$}}

% vecteurs
\newcommand*\colvec[3][]{
    \begin{pmatrix}\ifx\relax#1\relax\else#1\\\fi#2\\#3\end{pmatrix}
}

\pagestyle{fancy}

% ensembles N,Z,Q,D,R,C
\DeclareMathOperator{\NN}{\mathbb{N}}
\DeclareMathOperator{\ZZ}{\mathbb{Z}}
\DeclareMathOperator{\QQ}{\mathbb{Q}}
\DeclareMathOperator{\DD}{\mathbb{D}}
\DeclareMathOperator{\RR}{\mathbb{R}}
\DeclareMathOperator{\CC}{\mathbb{C}}
\DeclareMathOperator{\e}{e}
% droite D et plan P
\DeclareMathOperator{\GP}{\mathscr{P}}
\DeclareMathOperator{\GD}{\mathscr{D}}
\DeclareMathOperator{\GA}{\mathscr{A}}
\DeclareMathOperator{\GC}{\mathscr{C}}

\newcommand*{\QEDA}{\null\nobreak\hfill\ensuremath{\blacksquare}}
\newcommand*{\QEDB}{\null\nobreak\hfill\ensuremath{\square}}
\newcommand{\usd}{\frac{1}{2}}
\newcommand{\tsd}{\frac{3}{2}}
\newcommand{\csd}{\frac{5}{2}}
\newcommand{\usq}{\frac{1}{4}}
\newcommand{\tsq}{\frac{3}{4}}
\newcommand{\csq}{\frac{5}{4}}
\newcommand{\ssq}{\frac{6}{4}}
\newcommand{\psd}{\frac{\pi}{2}}

% angle deux vecteurs
\newcommand{\angv}[2]{(\widehat{\overrightarrow{#1},\overrightarrow{#2}})}
\newcommand{\vf}[1]{\overrightarrow{#1}}

% conjugué complexe
\newcommand{\zb}[1]{\overline{#1}}

\newcommand{\monNom}{Prénom Nom}
\newcommand{\maClasse}{T\textsuperscript{le}1}


\lhead{\monNom}
\chead{}
\rhead{\maClasse}

\lfoot{\jobname.tex}
\cfoot{}
\rfoot{\thepage}
\renewcommand{\headrulewidth}{0.4pt}
\renewcommand{\footrulewidth}{0.4pt}

\setlength{\droptitle}{-1.5cm}

\makeatletter
\renewcommand{\maketitle}{
  \thispagestyle{empty}
  \begin{center}
  \shadowbox{\parbox{4in}{%
     \centering%
     \textrm{\textbf{\Large \@title}}\\
     \vspace{0.2cm}
     \textrm{\large \@author}\\
     \vspace{0.2cm}
     \textrm{\large \@date}
  }} 
  \end{center}
  \null
}

\author{\monNom, \maClasse}
\title{Devoir maison x}
\date{\today} 

\begin{document}

\maketitle

\thispagestyle{fancy}

\section{Problème}

\subsection{Partie}

\subsubsection{}
Une équation
\begin{equation}
  \sqrt{x^2}=\lvert x\rvert \label{eq_abs}
\end{equation}

Une autre non numérotée:
\begin{equation*}
  \sum_{k=0}^nk^3=\left(\frac{n(n+1)}{2}\right)^2
\end{equation*}

Un système d'équations:
\begin{equation}
  \left\lbrace
  \begin{aligned}
    \binom{n}{k}      & = \frac{n!}{k!(n-k)!}                                                                       \\
    A_{m,n}           & =
    \begin{pmatrix}
      a_{1,1} & a_{1,2} & \cdots & a_{1,n} \\
      a_{2,1} & a_{2,2} & \cdots & a_{2,n} \\
      \vdots  & \vdots  & \ddots & \vdots  \\
      a_{m,1} & a_{m,2} & \cdots & a_{m,n}
    \end{pmatrix} \\
    \int_a^b f(x)\,dx & =\lim_{n\rightarrow+\infty}\frac{b-a}{n}\sum_{k=0}^{n} f\left(a+k\times\frac{b-a}{n}\right) \\
  \end{aligned}
  \right.
\end{equation}

Une référence à l'équation (\ref{eq_abs}).

Un blabla mathématique: $\forall x\in \RR^\star, \exists\, y\in \left]-1;1\right[\backslash \lbrace{0\rbrace}\, /\, x=\frac{1}{y} $

Ou encore $\forall \varepsilon>0, \exists N\in\mathbb{N}, \forall n\geqslant N, \left\vert u_n - l \right\vert < \varepsilon.$

Un vecteur 3D: $\overrightarrow{AB}\colvec[1]{-1}{-1}$ et un vecteur 2D: $\overrightarrow{CD}\colvec{-1}{-1}$. 

Un peu de géométrie: $\GD\cap\GP=\{A\}$, $(CD)\sslash \GD$.

De la géométrie et des complexes:
$\angv{OA}{OB}=\arg\left(\frac{b}{a}\right)=\arg(i)\equiv\frac{\pi}{2}[2\pi]$ ce qui montre que le triangle $OAB$ est rectangle en $O$.
$\omega=\e^{i\frac{2\pi}{5}}$, $\zb{\omega}=\e^{-i\frac{2\pi}{5}}$.

\subsubsection{}
Un tableau de signe: \\
\begin{center}
  \begin{tikzpicture}
    \tkzTabInit[espcl=2.0]
    {$x$ / 1 ,$f(x)$ /1 }%
    {$0$ , $\alpha$ , $+\infty$}%
    \tkzTabLine{ d, -, z , +, }
  \end{tikzpicture}
\end{center}

\subsection{Partie}
\subsubsection{}
Des limites avec surlignage du résultat:

\begin{equation*}
  \left.
  \begin{aligned}
    \lim_{x\rightarrow 0}\frac{\sin x}{x} & = 1      \\
    \lim_{x\rightarrow+\infty}\ln x       & =+\infty
  \end{aligned}
  \right\rbrace \text{par produit/somme, on en déduit donc que } \resm{\lim_{x\rightarrow+\infty}f(x)=+\infty}
\end{equation*}

\subsubsection{}
\begin{enumerate}[label=(\alph*)]
  \item sous-question
  \item autre sous-question
\end{enumerate}

\subsection{Partie}

\subsubsection{}\label{q2.A.3}
Un align pour une série d'équations:
\begin{align*}
  f^\prime(x) & =(x)^\prime \\
              & = \resm{1}  \\
\end{align*}
\section{Problème}

\subsection{Partie}

\subsubsection{}
Une référence à une question \ref{q2.A.3}.

Soustraction avec ligne:

\begin{align*}
  f(x)                            & = a+\cancel{b} \\
  - \;\;\;\;\;\;\;g(x)            & = \cancel{b}-a \\
  \cline{1-3}
  f(x)-g(x)                       & = a+a          \\
  \iff \;\;\;\;\;\;\;\;\;2f(x)    & = 2\times a    \\
  \iff \;\;\;\;\;\;\;\;\;\;\;f(x) & = a            \\
\end{align*}

\subsection{Partie}
\subsubsection{}

Un alignement plus complexe:

\begin{alignat*}{3}
       &   & f(x)=\frac{\e^x-\e^{-x}}{2} & >0       \\
  \iff &   & \e^x-\e^{-x}                & >0       \\
  \iff &   & \e^x                        & >\e^{-x} \\
  \iff &   & x                           & >-x      \\
  \iff &   & 2x                          & >0       \\
  \iff &   & x                           & >0
\end{alignat*}

\subsubsection{}

Tableau de variation:\\

\begin{tikzpicture}
  \tkzTabInit[espcl=6]{$x$/1,$f'(x)$/1, $f(x)$/2}{$-\infty$,$0$,$+\infty$}% 
  \tkzTabLine{ , , +, }%
  \tkzTabVar{- / $-\infty$,R / ,+ / $+\infty$ }
  \tkzTabVal[draw]{1}{3}{0.5}{0}{0}
\end{tikzpicture}

\subsubsection{}

Une figure géométrique en 3D: un cube.

\tdplotsetmaincoords{70}{20}
\usetikzlibrary{3d}
\begin{center}
\begin{tikzpicture}[scale=5, tdplot_main_coords]
\coordinate(A) at (0,0,0);
\coordinate(B) at (1,0,0);
\coordinate(C) at (1,1,0);
\coordinate(D) at (0,1,0);
\coordinate(E) at (0,0,1);
\coordinate(F) at (1,0,1);
\coordinate(G) at (1,1,1);
\coordinate(H) at (0,1,1);
\coordinate (I) at ($(A)!0.5!(B)$); % milieu [AB]
\draw[thick](A)--(B)--(F)--(E)--cycle;
\draw[thick](B)--(C)--(G)--(F)--cycle;
\draw[thick](E)--(F)--(G)--(H)--cycle;
\draw[dashed](A)--(D)--(C);
\draw[dashed](H)--(D);
\draw(A) node[anchor=north]{A};
\draw(B) node[anchor=north]{B};
\draw(C) node[anchor=west]{C};
\draw(D) node[anchor=east]{D};
\draw(E) node[anchor=east]{E};
\draw(F) node[anchor=north east]{F};
\draw(G) node[anchor=west]{G};
\draw(H) node[anchor=south]{H};
\draw(I) node{$\bullet$};
\draw(I) node[anchor=north east]{I};
\end{tikzpicture}
\end{center}

Une figure géométrique en 2D:

\usetikzlibrary{calc}
\begin{center}
\begin{tikzpicture}[scale=5] % 5cm pour 1cm
\coordinate(O) at (0,0);
\draw(O) node[anchor=north west]{O};
\draw[->,thick] (-1.3,0) -- (1.3,0) node (xaxis) [right] {$x$};
\foreach \x in {-1.2,-1.1,...,1.2} {
  %\draw (\x,0.02cm) -- (\x,-0.02cm) node[below] {$\x\phantom{-}\strut$};
  \draw (\x,0.02cm) -- (\x,-0.02cm);

}
\draw[->,thick] (0,-1.3) -- (0,1.3) node (yaxis) [above] {$y$};
\foreach \y in {-1.2,-1.1,...,1.2} {
  %\draw (0.02cm,\y) -- (-0.02cm,\y) node[below] {$\x\phantom{-}\strut$};
  \draw (0.02cm,\y) -- (-0.02cm,\y);

}
\coordinate(W) at (1,0);
\coordinate(J) at (0,1);
\coordinate(W1) at ({cos(72)},{sin(72)});
\coordinate(W2) at ({cos(144)},{sin(144)});
\coordinate(W3) at ({cos(216)},{sin(216)});
\coordinate(W4) at ({cos(288)},{sin(288)});
\draw(W) node{$\bullet$};
\draw(W) node[anchor=north west]{$\Omega_0$};
\draw(J) node{$\bullet$};
\draw(J) node[anchor=north east]{J};
\draw(O) circle(1);
\draw(W1) node{$\bullet$};
\draw(W1) node[anchor=south west]{$\Omega_1$};
\draw(W2) node{$\bullet$};
\draw(W2) node[anchor=south east]{$\Omega_2$};
\draw(W3) node{$\bullet$};
\draw(W3) node[anchor=north east]{$\Omega_3$};
\draw(W4) node{$\bullet$};
\draw(W4) node[anchor=north west]{$\Omega_4$};
\draw[thick,color=blue] (W)--(W1)--(W2)--(W3)--(W4)--cycle;
\end{tikzpicture}
\end{center}

Un exemple de géométrie euclidienne: la droite d'Euler:

\begin{center}
\begin{tikzpicture}

  % définitions des sommets du triange
  \def\Ax{0} \def\Ay{0}
  \def\Bx{9} \def\By{0}
  \def\Cx{2} \def\Cy{7}
  % longueur pour construire les médiatrices
  \def\d{2}

  \clip (-4,-4) rectangle (10,10);

  % triangle
  \coordinate[label=below left:$A$] (A) at (\Ax,\Ay);
  \coordinate[label=below right:$B$] (B) at (\Bx,\By);
  \coordinate[label=above:$C$] (C) at (\Cx,\Cy);
  \draw[name path=triangle,thick] (A) -- node[left] {$b$} (C) -- node[above] {$c$} (B) -- node[below] {$a$} cycle;
  
  % milieux & médianes
  \coordinate (IC) at ($(A)!0.5!(B)$);
  \coordinate (IB) at ($(A)!0.5!(C)$);
  \coordinate (IA) at ($(C)!0.5!(B)$);
  \draw[color=blue] (C) -- (IC);
  \draw[color=blue] (A) -- (IA);
  \draw[color=blue] (B) -- (IB);
  %\coordinate (G) at (barycentric cs:A=1,B=1 ,C=1);
  \coordinate (G) at (intersection of IA--A and IB--B);
  \node at (G){\color{blue}$\bullet$};
  \node[above] at (G){\color{blue}$G$};
  
  % côtés égaux
  \node[magenta] at ($(A)!0.5!(IC)$){$/$};
  \node[magenta] at ($(IC)!0.5!(B)$){$/$};
  
  % angles
  %\tkzMarkAngle[size=1cm,color=cyan,mark=|](C,A,B);
  
  % les angles
  %\tkzMarkAngle[size=0.5cm](B,A,C)
  %\tkzLabelAngle[pos=1](B,A,C){$\alpha$}
  %\tkzMarkAngle(C,B,A)
  %\tkzLabelAngle[pos=1.5](C,B,A){$\beta$}
  %\tkzMarkAngle[size=0.5cm](A,C,B)
  %\tkzLabelAngle[pos=1](A,C,B){$\gamma$}

  % hauteurs
  \tkzDefPointBy[projection=onto A--B](C)\tkzGetPoint{HC}
  \draw[color=red] (C) -- (HC);
  \tkzMarkRightAngle(B,HC,C)
  \tkzDefPointBy[projection=onto B--C](A)\tkzGetPoint{HA}
  \draw[color=red] (A) -- (HA);
  \tkzMarkRightAngle(C,HA,A)
  \tkzDefPointBy[projection=onto A--C](B)\tkzGetPoint{HB}
  \draw[color=red] (B) -- (HB);
  \tkzMarkRightAngle(A,HB,B)
  \coordinate(H)at(intersection of A--HA and B--HB);
  \draw(H) node{\color{red}$\bullet$};
  \draw(H)[above left] node{\color{red}$H$};
  
  % médiatrices
  % le point AM rotation de C de centre IA et d'angle 90 appartient à la médiatrice  
  \coordinate(AM)at([rotate around={90:(IA)}]C);
  \coordinate(BM)at([rotate around={90:(IB)}]A);
  \coordinate(CM)at([rotate around={90:(IC)}]A);
  % cercle circonscrit
  \coordinate(O)at(intersection of IA--AM and IC--CM);
  \node[right] at (O){\color{violet}$O$};
  \node[draw,color=violet,circle through=(A)]at(O){$\bullet$};
  % dessine les médiatrices
  \draw[color=violet] (IA)--(O);
  \tkzMarkRightAngle(O,IB,A)
  \draw[color=violet] (IB)--(O);
  \tkzMarkRightAngle(C,IA,O)
  \draw[color=violet] (IC)--(O);
  \tkzMarkRightAngle(O,IC,A)
  
  % droite d'Euler
  \draw[thick] (H)--(G)--(O);
    
  % bissectrices
  % cercle virtuel de centre B
  \path[name path=cB](B) circle (\d);
  % intersection cercle côtés
  \path[name intersections={of=cB and triangle, by={a,b}}];
  % milieu de la base du triangle isocèle
  \coordinate (Bbb) at ($1/2*(a)+1/2*(b)$);
  % pied de la bissectrice
  \coordinate (Bb) at (intersection of B--Bbb and A--C);
  \draw[color=green] (B)--(Bb);
  % cercle virtuel de centre C
  \path[name path=cC](C) circle (\d);
  % intersection cercle côtés
  \path[name intersections={of = cC and triangle, by={b,c}}];
  % milieu de la base du triangle isocèle
  \coordinate (Cbb) at ($1/2*(b)+1/2*(c)$);
  % pied de la bissectrice
  \coordinate (Cb) at (intersection of C--Cbb and B--A);
  \draw[color=green] (C)--(Cb);
  % cercle virtuel de centre A
  \path[name path=cA](A) circle (\d);
  % intersection cercle côtés
  \path[name intersections={of = cA and triangle, by={c,a}}];
  % milieu de la base du triangle isocèle
  \coordinate (Abb) at ($1/2*(c)+1/2*(a)$);
  % pied de la bissectrice
  \coordinate (Ab) at (intersection of A--Abb and C--B);
  \draw[color=green] (A)--(Ab);
  % centre du cercle inscrit:  {OM}=(AAb)n(BBb)n(CCb)
  \coordinate (OM) at (intersection of A--Ab and B--Bb);
  \draw(OM) node{\color{green}$\bullet$};
  \draw(OM)[above] node{\color{green}$\Omega$};
  % rayon du cercle inscrit: OMOA où OA est le projeté de OM sur [BC]
  \tkzDefPointBy[projection=onto A--B](OM)\tkzGetPoint{OC}
  \draw[color=magenta] (OM) -- (OC);
  \tkzMarkRightAngle(OM,OC,A)
  \tkzDefPointBy[projection=onto B--C](OM)\tkzGetPoint{OA}
  \draw[color=magenta] (OM) -- (OA);
  \tkzMarkRightAngle(OM,OA,B)
  \tkzDefPointBy[projection=onto C--A](OM)\tkzGetPoint{OB}
  \draw[color=magenta] (OM) -- (OB);
  \tkzMarkRightAngle(OM,OB,C)
  % égalité angles orientés
  \draw pic["$\alpha$", draw=gray, <->, angle eccentricity=1.3, angle radius=1.2cm] {angle=B--A--OM};
  \draw pic["$\alpha$", draw=gray, <->, angle eccentricity=1.3, angle radius=1.2cm] {angle=OM--A--C};
  \draw pic["$\beta$", draw=gray, <->, angle eccentricity=1.3, angle radius=1.2cm] {angle=C--B--OM};
  \draw pic["$\beta$", draw=gray, <->, angle eccentricity=1.3, angle radius=1.2cm] {angle=OM--B--A};
  \draw pic["$\gamma$", draw=gray, <->, angle eccentricity=1.3, angle radius=1.2cm] {angle=A--C--OM};
  \draw pic["$\gamma$", draw=gray, <->, angle eccentricity=1.3, angle radius=1.2cm] {angle=OM--C--B};
  
  % cercle inscrit
  \tkzCalcLength[cm](OM,OC)\tkzGetLength{r}
  %\draw(OM) let \p1 = ($ (OM) - (OC) $) in circle ({veclen(\x1,\y1)});
  \draw[color=green](OM) circle (\r);
  
\end{tikzpicture}
\end{center}

\subsubsection{}

Un algorithmes:

\begin{center}
\begin{minipage}{0.9\linewidth}
\SetCustomAlgoRuledWidth{13cm}
\setcounter{algocf}{1} % force le compteur à 2 pour l'algo
\begin{algorithm}[H]
\SetAlgoLined
\SetKwInOut{Input}{Entrée}
\SetKwInOut{Output}{Sortie}
\Input{$x_M, y_M, z_M, x_N, y_N, z_N, x_P, y_P, z_P$}
\Output{VRAI ou FAUX}
\BlankLine
$d \leftarrow x_N-x_M$\;
$e \leftarrow y_N-y_M$\;
$f \leftarrow z_N-z_M$\;
$g \leftarrow x_P-x_M$\;
$h \leftarrow y_P-y_M$\;
$i \leftarrow z_P-z_M$\;
$k \leftarrow d\times g+e\times h+f\times i$\;
$l \leftarrow d^2+e^2+f^2$\;
$m \leftarrow g^2+h^2+i^2$\;
\eIf{$l = m$ \bf{et} $k = 0$}{\Return{VRAI}}{\Return{FAUX}}
\caption{Détermine si le triangle $MNP$ est isocèle rectangle en $M$}
\end{algorithm}
\end{minipage}
\end{center}

\subsubsection{}

Une courbe:

\begin{figure}[!h]
\centering
\begin{tikzpicture}[scale=1]
  \def\xmin{-10} \def\xmax{10}
  \def\ymin{-1} \def\ymax{1}
  \def\nbpt{100}
  \begin{axis}[
    grid=both,
    xmin=\xmin,
    xmax=\xmax,
    ymin=\ymin,
    ymax=\ymax,
    xlabel=$x$,
    ylabel=$y$,
    axis lines={center},
    grid style={line width=.1pt,draw=lightgray},
    major grid style={line width=.2pt,draw=gray},
    minor tick num=4,
    enlargelimits={abs=0.1},
    tick style={draw=none},
    ticklabel style={font=\tiny},
    >=stealth
    ]
    \addplot[domain=\xmin:\xmax,blue,ultra thick,samples=\nbpt] plot[smooth] {sin(deg(x))/x} node [pos=0.9, above left] {$y=\frac{\sin(x)}{x}$};
    \addplot [domain=\xmin:\xmax,red,ultra thick,samples=500] {sin(deg(x))}
            node [pos=0.3, below right] {$y=x$};
    \addplot [domain=\xmin:\xmax,green,ultra thick,samples=500] {1/x}
            node [pos=1, below right] {$y=\frac{1}{x}$};
  \end{axis}
\end{tikzpicture}
\caption{Plusieurs courbes}
\end{figure}



\end{document}


